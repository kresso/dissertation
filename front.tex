%
%  revised  front.tex  2011-09-02  Mark Senn  http://engineering.purdue.edu/~mark
%  created  front.tex  2003-06-02  Mark Senn  http://engineering.purdue.edu/~mark
%
%  This is ``front matter'' for the thesis.
%
%  Regarding ``References'' below:
%      KEY    MEANING
%      PU     ``A Manual for the Preparation of Graduate Theses'',
%             The Graduate School, Purdue University, 1996.
%      TCMOS  The Chicago Manual of Style, Edition 14.
%      WNNCD  Webster's Ninth New Collegiate Dictionary.
%
%  Lines marked with "%%" may need to be changed.
%

  % Dedication page is optional.
  % A name and often a message in tribute to a person or cause.
  % References: PU 15, WNNCD 332.
\begin{dedication}
Dedication:

I would like to dedicate this work to my wife Julianne who has been such a support to me during my time at Purdue.  She has truly been the joy of my life and I am so grateful that I can be a part of her life.  I am also grateful for my children: Hyrum, Abraham, and Mercy.  They have been so patient with me and are always there ready to play with me whenever I can. Also to my parents Naomi and Kip.  They have always encouraged me to chase my dreams and provided me the encouragement and environment to do so.

I am grateful to Professor Bortoletto for being my advisor and allowing me the opportunity to join the CMS collaboration.  I am amazed at her love for physics and the dedication she puts into her work.  Also to those at Purdue that I have worked with, Jakub, Miguel, Petra, Daniele, and Mayra.  They have been such a support to me and have been so good to answer all my questions.

I would also like to thank all the members of the 2l2q group for all the work they have done.  Without them this analysis would never have been possible.


\end{dedication}

  % Acknowledgements page is optional but most theses include
  % a brief statement of apreciation or recognition of special
  % assistance.
  % Reference: PU 16.
%\begin{acknowledgments}
%  This is the acknowledgments.
%\end{acknowledgments}

  % The preface is optional.
  % References: PU 16, TCMOS 1.49, WNNCD 927.
%\begin{preface}
%  This is the preface.
%\end{preface}

  % The Table of Contents is required.
  % The Table of Contents will be automatically created for you
  % using information you supply in
  %     \chapter
  %     \section
  %     \subsection
  %     \subsubsection
  % commands.
  % Reference: PU 16.
\tableofcontents

  % If your thesis has tables, a list of tables is required.
  % The List of Tables will be automatically created for you using
  % information you supply in
  %     \begin{table} ... \end{table}
  % environments.
  % Reference: PU 16.
\listoftables

  % If your thesis has figures, a list of figures is required.
  % The List of Figures will be automatically created for you using
  % information you supply in
  %     \begin{figure} ... \end{figure}
  % environments.
  % Reference: PU 16.
\listoffigures

  % List of Symbols is optional.
  % Reference: PU 17.
%\begin{symbols}
%  $m$& mass\cr
%  $v$& velocity\cr
%\end{symbols}

  % List of Abbreviations is optional.
  % Reference: PU 17.
%\begin{abbreviations}
%  LHC& Large Hadron Collider\cr
%  CMS& Compact Muon Soelinoid\cr
%  SM& Standard Model\cr
%  fb$^{-1}$& inverse femtobarn\cr
%  HLT& High Level Trigger\cr
%  L1& Level 1 Trigger\cr
%  WP& Working Point\cr
%  POG& Physics Object Group\cr
%  PAG& Physics Analysis Group\cr
%  PDF& Probability Density Function\cr
%\end{abbreviations}

  % Nomenclature is optional.
  % Reference: PU 17.
%\begin{nomenclature}
%  Alanine& 2-Aminopropanoic acid\cr
%  Valine& 2-Amino-3-methylbutanoic acid\cr
%\end{nomenclature}

  % Glossary is optional
  % Reference: PU 17.
%\begin{glossary}
%  chick& female, usually young\cr
%  dude& male, usually young\cr
%\end{glossary}

  % Abstract is required.
  % Note that the information for the first paragraph of the output
  % doesn't need to be input here...it is put in automatically from
  % information you supplied earlier using \title, \author, \degree,
  % and \majorprof.
  % Reference: PU 17.
\begin{abstract}

%A search for a standard-model-like Higgs boson decaying into two Z bosons with ̄ is per-subsequent decay into two leptons and two quark-jets, H → ZZ → + − qq formed. The analysis uses 19.6 fb−1 of data collected by the CMS experiment from √ proton-proton collisions produced in LHC at s = 8 TeV. The analysis exploits the kinematic information and the flavor tagging of the leading particles of the event in order to isolate hypothetical Higgs boson signals with mass values in the range from 230 GeV/c2 to 650 GeV/c2 . No evidence of a Higgs boson signal is found and upper limits are set on the Higgs boson production cross section in that mass range.

  A search for a standard-model-like Higgs boson decaying into two $\PZ$ bosons with subsequent decay into two leptons and two quark-jets, H $\rightarrow$ ZZ $\rightarrow l^{+}l^{-} \Pq \Paq$, is presented. The CMS experiment collected 19.6 $\invfb$ of data in 2012 of $\Pp \Pp$ collisions at $\sqrt{s}$ = 8 TeV.  The analysis uses the kinematics of the final state and quark flavor tagging to select the Higgs boson signal in the mass range between 230 GeV and 650 GeV.  No evidence of a Higgs boson signal is found and upper limits are set on the Higgs boson production cross section in that mass range.

\end{abstract}
