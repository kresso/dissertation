%
%  summary.tex  2007-02-06  Mark Senn  http://www.ecn.purdue.edu/~mark
%

\chapter{Conclusion}

We have performed a search for a SM-like Higgs boson in the decay channel $H \rightarrow ZZ \rightarrow \LL \qqbar$.  This decay mode is relevant at high mass (200 to 650 GeV) and contributed significantly in a region that has not been probed before the LHC in any laboratory. The data samples analyzed are those collected by the CMS experiment in 2012 with an integrated luminosity corresponding to 19.6 $\invfb$ at 8 TeV, combined with those previously collected in 2011 with an integrated luminosity of 5.1 $\invfb$ at 7 TeV.

No evidence for a Standard Model Higgs boson has been found and the upper limits on the Higgs boson production cross section between 200 and 650 GeV have been computed.  In addition, the analysis excludes the existence of a standard-model-like Higgs boson in the mass range between 285 GeV to 650 GeV, where the expected exclusion range goes from 266 GeV to 626 GeV.

In the near future, this analysis will be extended in the high-mass region up to 1 TeV.  This analysis also sets a benchmark for other analysis and could be extended for the search of exotic, spin-2 particles, the Graviton, and other Standard Model Higgs-like particles.
