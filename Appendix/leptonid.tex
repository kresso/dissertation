

The  standard tag-\&-probe method used to evaluate the lepton identification efficiencies
from data requires the reconstruction of 
the dilepton system with invariant mass in the range [60-120]~\GeVcc.  One of the
leptons, called tag, is required to pass full selection criteria and to match the tighter
leg of the trigger.
The other lepton candidate, (the probe) is selected with criteria that depends on
the efficiency being measured.
The sample is divided into two exclusive subsamples depending on whether
the probe lepton passes or fails the selection criteria currently under 
investigation. Due to the presence of background events, 
the signal yields are obtained with a fit to the invariant mass distribution of the
dilepton system. The measured efficiency is calculated as a function of $p_T$ and $\eta$
of the probe lepton from the relative yields
of the signal in subsamples with passing or failing probes.
Finally, the data to Monte Carlo scale factors are deduced by dividing the efficiencies
in data by the ones obtained from Monte Carlo using exactly the same procedure.
Scale factors are used instead of raw efficiencies in order to benefit from 
partial cancellations of systematic uncertainites associated with the procedure.
The total efficiency measurement is factorized into five sequential relative
efficiency measurements: tracking, reconstruction, identification, isolation 
and the total trigger efficiency.  This can be expressed as following product:
\begin{equation*}
\label{eq:efficiency}
\epsilon_{lepton} = \epsilon_{tracking} \times \epsilon_{RECO/Tracking} \times \epsilon_{ID/RECO} \times \epsilon_{ISO/ID} \times \epsilon_{Trigger/ISO} 
\end{equation*}

Lepton identification requirements are given in Table~\ref{tab:electronid} for electrons,
and in Table~\ref{tab:muonid} for muons. The data to simulation scale factors for these
electron and muon identification requirements are listed in Table~\ref{eleSF}.

\begin{table}[htb]
\caption{Electron ID requirements for the Loose ID working point.}
\label{tab:electronid}
\vspace*{\medskipamount}
\begin{center}
\small
\begin{tabular}{|l|l|l|}
\hline
Variable & Barrel cut & Endcap cut  \\
\hline
$\Delta \eta_{trk, supercluster}$ & $<0.007$ & $<0.009$ \\
$\Delta \phi_{trk, supercluster}$ & $<0.15$ & $<0.1$ \\
$\sigma_{i\eta, i\eta}$ & $<0.01$ & $<0.03$ \\
$H/E$ & $<0.12$ & $<0.10$ \\
$d_{0}$ (wrt primary vertex) & $< 0.2 \mm$ & $< 0.2 \mm$ \\
$d_z$ (wrt primary vertex) & $< 2 \mm$ & $< 2 \mm$ \\
$|1/E - 1/p|$ & $< 0.05$ & $<0.05$ \\
$I_{PF, \,\, corr}/\pt$ & $< 0.15$ & $<0.15$ \\
Missing hits & $\le 1$ & $\le 1$ \\
Conversion vertex fit prob.& $< 10^{-6}$ & $< 10^{-6}$ \\
\hline
\end{tabular}
\end{center}
\end{table}

\begin{table}[htb]
\caption{Muon ID requirements for the Tight ID working point.}
\label{tab:muonid}
\vspace*{\medskipamount}
\begin{center}
\small
\begin{tabular}{|l|l|}
\hline
Variable & Cut  \\
\hline
isGlobalMuon & True \\
isPFMuon & True \\
%isTrackerMuon & True \\
$\chi^2 / ndof$ (global fit) & $<10$ \\
Muon chamber hits in global fit & $>0$ \\
Muon stations with muon segments & $>1$ \\
$d_{xy}$ (from tracker, wrt primary vertex) & $< 2 \mm$  \\
$d_z$ (from tracker, wrt primary vertex) & $< 5 \mm$  \\
Valid pixel hits (tracker track) & $> 0$ \\
Tracker layers with hits & $>5$ \\
$I_{PF, \,\, corr}/\pt$ & $< 0.12$ \\
\hline
\end{tabular}
\end{center}
\end{table}

