\chapter{Introduction}


The Standard Model of particles is one of the most successful theories that has been proposed. Since it's inception, the predictions given by this theory have been confirmed numerous times with an amazing amount of precision. This has happened in various fields and with a number of experimental techniques. Despite all of the successes of this theory, it's theoretical foundation rests on the observable quantum of the Higgs field called the Higgs boson. This particle lies as the capstone of the Standard Model. 

The Standard Model does not constrain the mass of the Higgs boson. In the past many experiments have searched for the Higgs boson, and while they were not successful in discovery, they were able to exclude its existence in a number of mass ranges. A lower bound of 114.4 GeV was set at the  Large Electron Positron (LEP) collider at CERN, and the Tevatron at Fermilab excluded the range of 156 - 177 GeV as well.

The Large Hadron Collider (LHC) was built at CERN to replace the LEP experiment. It is a superconducting proton-proton collider that ran at a center of mass energy of 7 TeV before 2012 and is running at 8 TeV currently. A large part of it's design has been to produce proof of the existence of the Higgs boson.

The Compact Muon Solenoid (CMS) experiment is one of the four major experiments that analyzes the collisions produced at the LHC. The detector is sensitive to a wide range of processes but has been created to be particularly successful in Higgs boson searches.

The Higgs boson is an unstable particle and so its detection depends on the Higgs boson decay products. If it has a sufficiently large mass, the decay of Higgs to vector bosons will dominate the branching ratio of the decay channels. Of particular importance is the production of a Z-boson pair. If one of the Z bosons decay into a lepton pair ($\Pe$ or $\Pgm$) then the possible backgrounds produced at hadron colliders can be significantly diminished by cuts on the sharp Z mass peak.

In this preliminary report we examine the feasibility of searching for the Higgs boson in the  H $\rightarrow$ ZZ $\rightarrow$ $l^{+}l^{-} \Pq \Paq$ decay channel with data collected by the CMS experiment. Jets in the final state create a both advantages and disadvantages.  Advantageously, the branching ratio of Z $\rightarrow$  $\Pq \Paq$ is significantly larger than Z $\rightarrow$  $l^{+}l^{-}$ thus improving the number of Higgs evens we can see.  The disadvantages of jets stem from complex reconstruction techniques and dealing with the added necessity of extremely effective background rejection.

% Standard Model
The second chapter of this preliminary report gives an introduction to the Standard Model and the theory behind the need for the Higgs boson.

% Experimental Apparatus
The third chapter describes the experiment, including both an introduction to the LHC and the CMS.  The various sub-detectors of the CMS experiment are also described.
% Reconstruction
%A description of the reconstruction algorithms used are given in chapter four.  This includes reconstruction for electrons, muons, and jets.

Chapter four describes the data-sets used for both data and Monte Carlo simulations.  Afterword a description of preselection cuts for the analysis is given.

%Evebt Selection
% 5 Event Selection

Trigger and Lepton Selection Efficiency for electrons are calculated and scale factors for data and Monte Carlo are given in Chapter five.


Chapter six gives the optimization for this signal channel.

%Systematics
%6 Systemnatics 


% Signal Optimization
%7 Signal Optimization


% Statistical Analysis
%8 Statistical Analysis
